\chapter{Zusammenfassung}
In diesem Kapitel wird die Arbeit kurz zusammengefasst und ein Überblick über
die nächsten Schritte gegeben.

\section{Fazit}
In dieser Arbeit wurde gezeigt, wie weit der Entwicklungsstand der gewählten
Graph-Bibliotheken ist. Dazu war es notwendig sich erst einmal über die
unterschiedlichen Konzepte zu informieren. Denn sowohl \enquote{Stream-Processing},
als auch Graphen sind für sich genommen schon sehr komplexe und teils neue Themen.
Dies gilt besonders in Bezug auf das Thema \gls{BigData}. Die Kombination von
beiden Themen ist dementsprechend nochmals komplexer, denn Konzepte, welche in
der eigenen Welt funktionieren, müssen in der kombinierten Welt nicht unbedingt
funktionieren. Denn es gibt verschiedene \enquote{Stream-Processing} Konzepte und
ein Graph besteht ebenfalls aus mehreren Konzepten. Dadurch ist auch klar, warum
es nicht die eine Bibliothek gibt, weil jede Bibliothek unterschiedliche Konzepte
gewählt hat und diese testen muss, ob sie für einen produktiven Einsatz geeignet
sind.

Zum Zeitpunkt der Arbeit waren alle Bibliothek noch in einem experimentellen
Status. Dies wird vermutlich noch eine Weile so bleiben. Denn obwohl alle
Bibliothek über gute Ansätze verfügen, mangelt es vor allem noch an ausreichendem
Einsatzgebieten und den daraus resultierenden Testfällen, Daten und Beispielen.
Die derzeit vorhandenen Beispiele gehen mehr oder weniger nicht über ein normales
\enquote{Hello World} hinaus. Obwohl \enquote{gelly-streaming} durch Apache Flink
schon über einen guten Unterbau verfügt. Des Weiteren scheint die Entwicklung bei
allen mehr oder weniger zu stagnieren. Dies kann natürliche unterschiedliche
Ursachen haben. Zunächst einmal alle Bibliothek wurden als Pionierarbeit an
Universitäten entwickelt. Diese ist schon mal ein enormer Aufwand, der im
Allgemeinen auch nicht entlohnt wird. Ausgehend von dieser Pionierarbeit dann
eine Firma zu gründen bzw. eine Firma zu finden, welche dies weiterentwickelt ist
nicht gerade einfach, da die Projekte meistens noch nicht wirklich ausgereift
sind. Die Firmen müssen jedoch ihre Finanzen im Blick behalten, denn sie haben
ja Angestellte und diese wollen auch noch in einigen Jahren ihren Lohn
bekommen. Ein Projekt zu finanzieren, welches dann vielleicht nach einiger
Entwicklungszeit von den Projektbeteiligten eingestellt werden muss ist, ist da
selten zu verkraften. Bei kleinen hausinternen Projekten ist, dies meistens
noch eher zu verkraften. Auf der anderen Seite, wenn es sich um eine
revolutionäre Idee handelt, welche einzigartig und von den Firmen dringen gesucht
wird, wird dies selten als Open-Source-Projekt zu Ende gebracht. Denn diese
Projekte sind meistens die Einstiege in neue Jobs. Ein weiterer Punkt ist natürlich,
dass die aktiven Entwickler sich anderen Projekten bzw. Aufgaben zu wenden können.
Dies wird gerade auch bei \enquote{gelly-streaming} klar, wo derzeit keine bis
nur eine minimale Entwicklung stattfindet, der Unterbau Apache Flink wird
im Gegensatz dazu jedoch sehr stark weiterentwickelt. Wie die weitere Entwicklung
von Apache Flink weitergeht, lässt sich nicht vorhersagen. Dies gilt ins
besondere, da die Firma \enquote{data Artisan}, welche die Entwicklung von
Apache Flink betreibt, von \enquote{Alibaba} gekauft wurde.

Das Beispiel konnte nicht mit allen Bibliotheken praktisch umgesetzt werden. Dies
hängt vor allem mit der zum Teil schlechten Infrastruktur und dem experimentellen
Status zusammen, denn die Entwürfe für die einzelnen Bibliotheken wurden ja
entwickelt. Alle Bibliotheken nutzen sehr verschiedene Ansätze und haben noch
einige Probleme bzw. müssen sich noch weiter bewähren.

Deshalb gibt es auch keinen klaren Favoriten. Wenn jedoch Apache Flink in die
Betrachtung mit einfließt, dann hat \enquote{gelly-streaming} schon einen klaren
Vorteil. Zumal Apache Flink mit \enquote{Gelly} schon eine Graph-Bibliothek für
\enquote{Batch-Processing} hat und es angedacht ist, dass diese mit
\enquote{gelly-streaming} zusammengeführt wird. Die Bibliothek \enquote{Table API \& SQL}
baut sowohl auf der \enquote{Stream-Processing}- wie auch auf der
\enquote{Batch-Processing}-\gls{API} von Apache Flink auf. Es ist jedoch klar
anzumerken, dass keine der hier getesteten Bibliotheken für einen produktiven
Einsatz geeignet sind.

\section{Ausblick}
Für die nächsten Entwicklungsschritte bieten sich mehrere Möglichkeiten an. Zum
einen könnte die gesamte Infrastruktur von \enquote{Gephi} aktualisiert werden.
Dadurch würde eine weitere Bibliothek zum Testen hinzukommen bzw. die zugrunde
liegenden Ideen könnten überprüft und gegebenenfalls für andere Projekte benutzt
werden.

Eine andere Möglichkeiten besteht darin, für die beiden anderen Bibliotheken ein
größeres Beispiel zu definieren und dieses dann praktisch umzusetzen. Um
anschließend genau zu wissen, welche Bibliothek für welchen Einsatzzweck geeignet
ist und welche nicht weiter entwickelt werden sollte. Eine direkte Weiterentwicklung
von \enquote{gelly-streaming} ist jedoch abzuraten, da diese eine direkte Beziehung
zu Apache Flink hat und es ist deshalb anzuraten sich mit den Entwicklern von
Apache Flink auseinander zu setzen.
