\begin{abstract}
Die Arbeit beschäftigt sich mit dem Thema \enquote{Stream-Processing} von
Graphen. Ziel der Arbeit ist es die gewählten Bibliotheken hinsichtlich ihrer
Funktionalität zu analysieren. Des Weiteren sollen die praktischen Möglichkeiten
ermittelt werden.

Dazu wird in den ersten Kapiteln die Theorie für die einzelnen Aspekte gelegt.
Dabei wird zuerst die Datenstruktur eines Graphen erläutert. Anschließend wird
das Konzept \enquote{Stream-Processing} vorgestellt. Abschließend werden die
einzelnen Bibliotheken vorgestellt und miteinander verglichen.

Nachdem die theoretischen Grundlagen gelegt wurden, wird das praktische Beispiel
erläutert und für die jeweiligen Bibliotheken ein Konzept entworfen und diesen
anschließend praktisch umgesetzt.

Abschließend erfolgt eine Zusammenfassung der Arbeit und ein Ausblick auf die
möglichen nächsten Schritte.
\end{abstract}

\begin{english}
    \begin{abstract}
        The topic of the thesis is \enquote{Stream-Processing} of graphs.
        The goal of the thesis is to analyses the functionality of the selected
        libraries. That includes the research of how useable they are in practice.
        
        In the first chapters the theory for every aspect will be explained.
        First the graph datastructure is described. After that the concept and
        methods of \enquote{Stream-Processing} are explained. At the end of
        chapter the different libraries are presented and compared.
        
        After the theory, the practical example is explained. That means that
        for every library a design is created and implemented.
        
        At the end of the thesis the summary of the work and a preview of the
        next possible steps are described.
    \end{abstract}
\end{english}
