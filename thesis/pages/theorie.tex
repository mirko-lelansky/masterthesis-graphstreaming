\chapter{Theorie}
In diesem Kapitel werden die theoretischen Grundlagen für die Arbeit beschrieben.

Dabei werden zuerst die verschiedenen Graphen erläutert. Anschließend wird der
Stream-Prozess beschrieben, mit den verschiedenen Modellen.

Danach folgt eine Vorstellung der verschiedenen Bibliotheken und welche
zugrundeliegende Streaming-Modelle dort zum Einsatz kommen. 

\section{Graphen}
Ein Graph G ist eine Datenstruktur mit der folgenden Definition: $G = (V,E)$.
Dabei ist V die Menge von Knoten und E die Menge von Kanten. Jede Kante besteht
dabei aus einen Paar von V. Bei einem ungerichteten Graphen ist das Paar, welches
die Kante representiert, ein ungeordnetes Paar oder andersgesagt eine
zweielementige Menge von V. Im Gegensatz dazu ist es bei einem gerichteten
Graphen ein geordnetes Paar.

\blockquote[\cite{Aigner2015}]{
[\dots]Ein Graph $G = (V,E)$ besteht aus einer endlichen Menge V von Ecken,
einer endlichen Menge E von Kanten, und einer Vorschrift, welche jeder Kante e
genau zwei (verschiedene oder gleiche) Ecken a und b zuordnet, die wir die
Endecken von e nennen. Normalerweise sind die Endecken a und b verschieden;
ist a = b, so nennen wir e eine Schlinge bei a. Hat e die Endecken a und b, so
sagen wir, e verbindet a und b.[\dots]}

Wenn die beide Elemente der Kante identisch sind spricht man auch von einer
Schlinge. Gibt es mehrere identische Kanten spricht man von Merhfachkanten.
Diese lassen sich zusammenfassen in dem nur eine Kante dargestellt mit einer
zusätzlichen Zahl, welche die Mehrfachkanten representiert.

Es werden zwei Klassen von Graphen unterschieden. Grpahen, welche nur keine
Schlingen und Mehrfachkanten zulassen, werden als einfache Graphen bezeichnet,
sonst als Multigraphen.

\blockquote[\cite{Gurski2010}]{
[\dots]Kanten, die mit mehr als zwei Knoten inzident sind, werden Hyperkanten
genannt. Graphen mit Hyperkanten heißen Hypergraphen. Wir betrachten in diesem
Buch jedoch vorwiegend einfache Graphen, also Graphen ohne multiple Kanten und
ohne Schleifen, und auch keine Hypergraphen.}

\section{Stream-Processing}
Stream-Processing beschreibt einen Prozess, bei dem die kontinuierlich
ankommenden Daten verarbeitet bzw. transformiert werden. Ein Stream ist eine
unendliche Menge von Elementen. Die Daten werden dabei von mindestens einer
Quelle gelesen. Anschliend werden diese von mindestens einer Verarbeitungseinheit
transformiert und abschließend in mindestens ein Ziel geschrieben. Quelle und
Ziel sind dabei externe Systeme, wie Datenbanken, Messaging-Platformen,~\dots.

Streaming-Modelle sind die zugrundeliegenden Modelle, der
jeweiligen Verarbeitungseinheit. Diese Modelle haben mehrere Aufgaben. Zunächst
definieren diese Modelle, wie ein Graph bzw. Teilgraph dargestellt wird.

%% Graph-Typen

\section{Graph-Streaming Bibliotheken}
Graph-Streaming Bibliotheken sind Bibliotheken, welche die beiden Welten \gls{BigData}
und Graphen miteinander kombinieren wollen. Dabei geht es darum große Graphen
effizient und möglichst in Echtzeit mit Hilfe von Streams zu verarbeiten. Die
bedeutet, dass die Daten, welche von einem Stream verarbeitet werden, keine
einfachen Zeichenketten mehr sind, sondern zum Beispiel eine Kante des Graphens.
Je nach dem welches Streaming-Modell zum Einsatz kommt, können auch andere Daten
hinzu kommen. In dieser Arbeit werden die Bibliotheken \enquote{gelly-streaming},
\enquote{graphstream-project} und \enquote{Gephi} verglichen.

\subsection{gelly-streaming}
\enquote{gelly-streaming} ist eine Graph-Streaming Bibliothek für Apache Flink,
welche von Vasiliki Kalavri am KTH in Schweden im Jahr 2015 entwickelt wurde.
Apache Flink eine verteielte Platform für Batch- und Stream-Processing. Der Kern
von Apache Flink ist die Streaming-Engine, welche die eigentliche Verarbeitung
der Daten vornimmt. Alle \glspl{API} und Bibliotheken bauen auf der Engine auf.
Apache Flink wurde an der TU Berlin entwickelt und ist jetzt ein Apache Projekt.
Apache Flink ist in der Programmiersprache Java geschrieben. Es existiert jedoch
auch eine \gls{API} für Scala, welche nur ein Wrapper für Java darstellt. Dadurch
können Entwickler ihre Programme in Java oder Scala schreiben. Für alle anderen
Java-Alternativen müssen die benötigten Laufzeitbibliotheken mitgeliefert
werden. Die aktuelle Version von Flink ist 1.6.0 .

Bei der Bibliothek \enquote{gelly-streaming} kommt ein semi-streaming Verfahren
zum Einsatz, da die Bibliothek für Apache Flink entwickelt wurde. Dabei wird
jeweils immer eine Kante mit den beiden Knoten gespeichert. Jedoch keine
Zusatzinformationen, wie zum Beispiel ob die Kante hinzugefügt oder entfernt
wurde. Dies muss vom Anwender sichergestellt werden. \enquote{gelly-streaming}
ist ebenfalls in Java geschrieben und wird vorrangig von Studenten von
Vasiliki Kalavri weiterentwickelt. Allerdings ist anzumerken, dass die Bibliothek
zum Zeitpunkt der Arbeit noch im experimentellen Stadium ist und deshalb noch
keine Alpha/Beta/Finale-Versionen zur Verfügung stehen.

%% Bild mit gelly-streaming

\subsection{graphstream-project}
\enquote{graphstream-project} ist eine Java-Bibliothek für Graph-Streaming. Die
Bibiliothek wurde 2009 von der Gruppe RI\textsubscript{2}C-Gruppe am LITIS ein
Zusammenschluss von mehreren Universitäten und Firmen entwickelt. Derzeit wird
die Bibliothek von der Universität Le Havre weiterentwickelt. Die aktuelleste
Version ist 1.3 .


%% Streaming-Modell
%% Architektur

\subsection{Gephi}
\enquote{Gephi} ist eine Visualisierungsplatform für Graphen, welche in Java
geschrieben ist. Diese wurde 2009 an der Universität of Technologie of Compiègne
in Frankreich entwickelt. Derzeit wir die Platform vom Gehpi Consortium betreut.
Die aktuelleste Version ist 0.9.2 .

%% Streaming-Modell
%% Architektur
