\chapter{Theorie}
In diesem Kapitel werden die theoretischen Grundlagen für die Arbeit beschrieben.

Dabei werden zuerst die verschiedenen Modelle beschrieben, welche beim Streaming
zum Einsatz kommen.

Anschlieend werden die Graph-Streaming Bibliotheken beschrieben und welche
Streaming-Modelle dort zum Einsatz kommen.

Abschließend folgt ein Vergleich der verschiedenen Systeme anhand des
Kriterienkataloges.

\section{Streaming-Modelle}
\section{Graph-Streaming Bibliotheken}
Graph-Streaming Bibliotheken sind Bibliotheken, welche die beiden Welten \gls{BigData}
und Graphen miteinander kombinieren wollen. Dabei geht es darum große Graphen
effizient und möglichst in Echtzeit mit Hilfe von Streams zu verarbeiten. Die
bedeutet, dass die Daten, welche von einem Stream verarbeitet werden, keine
einfachen Zeichenketten mehr sind, sondern zum Beispiel eine Kante des Graphens.
Je nach dem welches Streaming-Modell zum Einsatz kommt, können auch andere Daten
hinzu kommen. In dieser Arbeit werden die Bibliotheken \enquote{gelly-streaming},
\enquote{graphstream-project} und \enquote{Gephi} verglichen.

\subsection{gelly-streaming}
\enquote{gelly-streaming} ist eine Graph-Streaming Bibliothek für Apache Flink,
eine verteielte Platform für Batch- und Stream-Processing. Der Kern von
Apache Flink ist die Streaming-Engine, welche die eigentliche Verarbeitung der
Daten vornimmt. Alle \glspl{API} und Bibliotheken bauen auf der Engine auf.

%% flink.adoc
%% gelly Entwicklung
\subsection{graphstream-project}
\subsection{Gephi}
