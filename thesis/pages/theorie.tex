\chapter{Theorie}
In diesem Kapitel werden die theoretischen Grundlagen für die Arbeit beschrieben.

Dabei werden zuerst die verschiedenen Graphen erläutert. Anschließend wird der
Stream-Prozess mit den verschiedenen Modellen beschrieben.

Danach folgt eine Vorstellung der verschiedenen Bibliotheken und welche
zugrundeliegende Streaming-Modelle dort zum Einsatz kommen. 

\section{Graphen}
Ein Graph G ist eine Datenstruktur mit der folgenden Definition: $G = (V,E)$.
Dabei ist V die Menge von Knoten und E die Menge von Kanten. Jede Kante besteht
dabei aus einen Paar von V. Bei einem ungerichteten Graphen ist das Paar, welches
die Kante representiert, ein ungeordnetes Paar oder andersgesagt eine
zweielementige Menge von V. Im Gegensatz dazu ist es bei einem gerichteten
Graphen ein geordnetes Paar.

\textquote[\cite{Aigner2015}]{
[\dots] Ein Graph $G = (V,E)$ besteht aus einer endlichen Menge V von Ecken,
einer endlichen Menge E von Kanten, und einer Vorschrift, welche jeder Kante e
genau zwei (verschiedene oder gleiche) Ecken a und b zuordnet, die wir die
Endecken von e nennen. Normalerweise sind die Endecken a und b verschieden;
ist a = b, so nennen wir e eine Schlinge bei a. Hat e die Endecken a und b, so
sagen wir, e verbindet a und b. [\dots]
}

Wenn die beide Elemente der Kante identisch sind spricht man auch von einer
Schlinge. Gibt es mehrere identische Kanten spricht man von Merhfachkanten.
Diese lassen sich zusammenfassen in dem nur eine Kante dargestellt mit einer
zusätzlichen Zahl, welche die Mehrfachkanten representiert.

Es werden zwei Klassen von Graphen unterschieden. Grpahen, welche nur keine
Schlingen und Mehrfachkanten zulassen, werden als einfache Graphen bezeichnet,
sonst als Multigraphen.

\textquote[\cite{Gurski2010}]{
[\dots] Kanten, die mit mehr als zwei Knoten inzident sind, werden Hyperkanten
genannt. Graphen mit Hyperkanten heißen Hypergraphen. Wir betrachten in diesem
Buch jedoch vorwiegend einfache Graphen, also Graphen ohne multiple Kanten und
ohne Schleifen, und auch keine Hypergraphen.
}

\section{Stream-Processing}
Stream-Processing beschreibt einen Prozess, bei dem die kontinuierlich
ankommenden Daten verarbeitet bzw. transformiert werden. Ein Stream ist eine
unendliche Liste von Elementen. Die Daten werden dabei von mindestens einer
Quelle gelesen. Anschliend werden diese von mindestens einer Verarbeitungseinheit
transformiert und abschließend in mindestens ein Ziel geschrieben. Quelle und
Ziel sind dabei externe Systeme, wie Datenbanken, Messaging-Platformen,~\dots,
welche in der Regel verschieden sind.

Die Idee des Stream-Processing ist schon sehr alt und im Laufe der Zeit sind
mehrere Modelle entstanden. Jedes Modell hat dabei gewisse Vorraussetzungen
und Einschränkungen, welche direkten Einfluss auf die konkreten Umsetzungen
haben. Es gibt die Modelle \enquote{Classical Streaming}, \enquote{semi-Streaming},
\enquote{W-Stream} und \enquote{StreamSort}.

\subsection{Classical Streaming}
Das erste Modell ist das \enquote{Classical Streaming}. Dieses wurde in den
1980 von Munro und Paterson definiert. Das Modell beschreibt die Verarbeitung
eines Streams durch eine \gls{RAM}-Maschine. Der Stream ist dabei eine Folge von
Zeichen aus einem definierten Alphabet, welche sequenziell verarbeitet werden
können.

Die Prameter dieses Modells sind der Speicher der \gls{RAM}-Maschine in Bits und
die Anzahl der Verarbeitungsdurchläufe des Streams. Beide Parameter sind dabei
Abhängig von der Länge des Streams und sollen möglichst klein im Verhältnis zur
Länge des Streams sein.

\foreigntextquote{english}[\cite{Ribichini2007}]{
In classical streaming, input data can be accessed sequentially in the form
of a data stream, and need to be processed using a working memory that
is small compared to the length of the stream. The main parameters of the
model are the number p of sequential passes over the data and the size s of the
working memory (in bits). [\dots]
}

Das Modell macht dabei keine Aussagen über die Laufzeit des Algorithmus. Diese
spielt für viele Probleme jedoch eine Rolle, deshalb wurde das
\enquote{semi-Streaming} Modell entwickelt.

\foreigntextquote{english}[\cite{Ribichini2007}]{
Notice that our definition imposes no restrictions on the amount
of computation performed by the RAM machine, as it is often the case when
dealing with external memory models, where it is assumed that I/O operations
take orders of magnitude longer than internal memory operations. There
are, however, applications in which the per-item processing time (average,
maximum) is a significant parameter that should be taken into account.
}

\subsection{semi-Streaming}
Das \enquote{semi-Streaming} Modell ist ein vereinfachtes Modell des
\enquote{Classical Streaming} Modells. Im Gegensatz zum \enquote{Classical Streaming}
wird hier festgelegt, das die Anzahl der Durchläufe beschränkt wird auf eins bzw.
einige wenige. Des Weiteren wird die Speichergröße der RAM-Maschine ebenfalls
auf beschränkt auf die logarithmische-Komplexitätsklasse.

\foreigntextquote{english}[\cite{Ribichini2007}]{
In particular, some recent papers show that several graph problems can be
solved with one or few passes in the Semi-streaming model [53] where the
working memory size is $O(n · \text{polylog} n)$ for an input graph with n
vertices (or even $O(n^{1 + \epsilon})$, with $\epsilon < 1$, for applications
like spanners, for which linear memory in the number of vertices is provably not
sufficient): [\dots].
}

Diese Beschränkungen sorgen dafür, dass die Laufzeit des Algorithmus verbessert
wird. Jedoch haben diese Beschränkungen auch den Nachteil, dass sie nur
näherungsweise Ergebnisse liefern.

\foreigntextquote{english}[\cite{Ribichini2007}]{
Despite the heavy restrictions of the classical streaming model, major
success has been achieved for several data sketching and statistics problems,
e.g., approximate frequency moments [7], histogram maintenance [58],
L\textsuperscript{1} difference [55], where $O(1)$ passes and polylogarithmic
working space have proven enough to find approximate solutions (see also the
bibliographies in [14, 88]).
}

\subsection{W-Stream}
Das \enquote{W-Stream} Modell oder \enquote{Write-Stream} Modell ist eine
Erweiterung des \enquote{Classical Streaming}. Dabei wird jedes gelesene Element
des Streams nach der Verarbeitung in einen Ausgabe-Stream geschrieben. Die
Elemente können dabei bevor sie in den Ausgabe-Stream landen, verändert werden.
Wenn die Elemente nicht verändert werden, handelt es sich um einen speziellen
\enquote{W-Strem} nämlich den \enquote{Classical Stream}.

\foreigntextquote{english}[\cite{Ribichini2007}]{
In the W-Stream model [97], a streaming pass, while reading data from
the input stream and processing them in the working memory, produces items
that are sequentially appended to an output stream. [\dots]
Clearly, $\text{Stream}(p, s) \subseteq \text{W-Stream}(p, s)$ since at every
s/w-pass the output stream may simply consist of a copy of the input stream.
}

\subsection{StreamSort}
Das \enquote{StreamSort} Modell ist eine Erweiterung zum \enquote{W-Stream}.
Dabei schließt sich nach dem Schreibprozess noch ein Sortierungsprozess an. Dabei
werden die Elemente noch einer vorgegebenen globalen Sortierungsreihenfolge
sortiert.

\foreigntextquote{english}[\cite{Ribichini2007}]{
[\dots] This model extends classical streaming in two ways the ability to write
intermediate temporary streams and the ability to reorder them at each pass for
free. [\dots]
}

\section{Graph-Streaming Bibliotheken}
Graph-Streaming Bibliotheken sind Bibliotheken, welche die beiden Welten
\gls{BigData} und Graphen miteinander kombinieren wollen. Dabei geht es darum
große Graphen bzw. Teilgraphen effizient und möglichst in Echtzeit mit Hilfe von
Streams zu verarbeiten.

Bevor eine Verarbeitung der Daten stattfinden kann, muss die Representation
definiert sein. In unserem Fall sollen Streams Verarbeitet werden, deren Daten
unseren Graphen representieren. Um den Graphen darzustellen gibt es nun mehrere
Möglichkeiten.

Die erste Möglichkeit ist es zwei seperate Streams zu verwenden einen für die
Knoten und eine für die Kanten. Um eine Verarbeitung durchzuführen ist es bei
diesem Modell notwendig, die beiden Streams zu kombinieren.

\foreigntextquote{english}[\cite{Bali2015}]{
Combining edge and vertex streams The first approach was to have separate
edge and vertex streams. When we need to know the vertex values corresponding
to the end points of an edge, the two streams can be joined.
}

Eine andere Möglichkeit ist es die Werte der beiden Knoten und die dazugehörige
Kante mit abzuspeichern. Das Problem bei dieser Darstellung ist allerdings, 
falls nicht alle Daten der Knoten vorhanden sind, entstehen Fehler.

\foreigntextquote{english}[\cite{Bali2015}]{
Triplet stream Another way to stream graphs would be the use of triplets.
Triplets are pairs of vertices around an edge. Both the edge and the two vertices
can have their own value.
}

Die einfachste Möglichkeit besteht darin einfach ein Stream aus Knotenpaaren zu
nehmen, welche genau eine Kante representieren.

\foreigntextquote{english}[\cite{Bali2015}]{
Edge-only stream The final model we considered is the simplest. The graph
consists of a single stream of edges.
}

Allerdings gibt es bei allen Darstellungsformen einige Probleme. Zunächst einmal
gibt es keine Information, ob es sich um einen ungerichtete oder gerichtete
Kante handelt. Des Weiteren existiert auch keine Statusinformation über die
Änderung der Kante bzw. über den Grund was zum erzeugen der Kante geführt hat
z.B. wurde die Kante oder ein Knoten hinzugefügt, gelöscht, geändert,~\dots .
Ein anderes Problem tritt auf, wenn nur ein Knoten hinzugefügt wird, wie wird
dieser dann dargestellt bzw. verarbeitet. Außerdem stellt sich die Frage, wie
gewichtete Graphen dargestellt werden, welche bei der Routenplanung zum Einsatz
kommen.

Die Funktionalität der Bibliotheken richtet sich auch nach dem zugrundeliegende
Streaming-Modell bzw. der Laufzeitumgebung. In dieser Arbeit werden die
Bibliotheken \enquote{gelly-streaming}, \enquote{graphstream-project} und
\enquote{Gephi} verglichen.

\subsection{gelly-streaming}
\enquote{gelly-streaming} ist eine Graph-Streaming Bibliothek für Apache Flink\footnote{\url{https://flink.apache.org}}.
Diese wurde von Vasiliki Kalavri am KTH in Schweden im Jahr 2015 entwickelt.
Sie ist in der Programmiersprache Java geschrieben und besitzt auch kein andere
Schnittstelle. Derzeit ist die Bibliothek noch in einem experimentellen Stadium,
deshalb gibt es noch keine Alpha/Beta/Finale-Versionen. Die Bibliothek verwendet,
dass semi-Streaming Modell. Bei dem nur der Stream aus Kanten abgespeichert wird.
Die Kante wird dabei durch zwei Knoten representiert. Die Bibliothek ist dabei
in ihrer Funktionalität gebunden, an ihre Laufzeitumgebung
Apache-Flink, deren \gls{API} sie benutzt. Derzeit wird Apache 1.2.0 benutzt.

\foreigntextquote{english}[\cite{Bali2015}]{
The semi-streaming model allows for the storage of O(polylog n) elements for n
edges, and also O(polylog n) passes. Since we cannot have multiple passes in our
model, we use a modified semi-streaming model. Not every algorithm of the
semi-streaming model will be possible to implement in our model.
}

Apache Flink eine verteielte Platform für Batch- und Stream-Processing. Der Kern
von Apache Flink ist die Streaming-Engine, welche die eigentliche Verarbeitung
der Daten vornimmt. Apache Flink wurde in der Programmiersprache Java an der TU
Berlin entwickelt und ist jetzt ein Apache Projekt. Zusätzlich zur Java-\gls{API}
existiert noch eine Scala-\gls{API}, welche als Wrapper dient. Die aktuelleste
Version ist 1.6.0. 

Apache Flink ist modular aufgebaut. Die Architektur ist im Bild \ref{fig:flink-architecture}
dargestellt. Auf die Streaming-Engine bauen die beiden \gls{API} auf.
Die \enquote{DataStream-API} definiert die verschiedenen Schnittstellen und Klassen
für Streams. Die \enquote{DataSet-API} definiert die verschiedenen Schnittstellen
und Klassen für Datenfelder bzw. Listen. Diese werden dabei von der Engine in
endliche Streams umgewandelt.

\begin{figure}
\centering
\includegraphics[height=10cm, width=15cm]{../material/images/flink-stack-graphic.png}
\caption{Architektur von Apache Flink \cite{Foundation2018}}
\label{fig:flink-architecture}
\end{figure}

\foreigntextquote{english}[\cite{Foundation2018}]{
The basic building blocks of Flink programs are streams and transformations.
(Note that the DataSets used in Flink’s DataSet API are also streams internally
– more about that later.) Conceptually a stream is a (potentially never-ending)
flow of data records, and a transformation is an operation that takes one or
more streams as input, and produces one or more output streams as a result.

When executed, Flink programs are mapped to streaming dataflows, consisting of
streams and transformation operators. Each dataflow starts with one or more
sources and ends in one or more sinks. The dataflows resemble arbitrary
directed acyclic graphs (DAGs). Although special forms of cycles are permitted
via iteration constructs, for the most part we will gloss over this for
simplicity.
}

\subsection{graphstream-project}
\enquote{graphstream-project} ist eine Java-Bibliothek für Graph-Streaming. Die
Bibiliothek wurde 2009 von der Gruppe RI\textsubscript{2}C-Gruppe am LITIS ein
Zusammenschluss von mehreren Universitäten und Firmen entwickelt. Derzeit wird
die Bibliothek von der Universität Le Havre weiterentwickelt. Die aktuelleste
Version ist 1.3 .

Die Bibliothek besteht aus drei verschiedenen Modulen core, algo und ui. Das core
Modul enthält alle Basisklassen zum lesen und anzeigen von Graphen. Das algo
Modul enthält zusätzliche Algorithmen für Graphen. Das ui Modul enthält ein
alternatives Framework zur Darstellung der Graphen.

\foreigntextquote{english}[\cite{Team2018}]{
gs-core is the base of GraphStream which is all you need to read a graph and
display it easily. gs-algo contains extra algorithms which can be run on a graph.
gs-ui provides other graphic viewers and will contain sources for graphical tools.
}

Das Bild zeigt einen beispielhaften Ablauf einer Verarbeitungsprozesses. Der
Graph wird aus einer Datei gelesen und die erzeugten Events stellen das Ziel dar.
Diese Events sind gleichzeitig die Quelle der Verarbeitungseinheit, welche
wiederum neue Events erzeugt. Diese neuen Events stellen die Quelle für die
Ausgabe dar. Das verwendete Streaming-Modell wird dabei nicht erwähnt. Aufgrund
der Tatsache, dass die Bibliothek sowohl Einfache- als auch Multigraphen abdecken
kann, ist es aber warhscheinlich, dass \enquote{Classical Streaming} zum Einsatz
kommt.

\foreigntextquote{english}[\cite{Team2018}]{
GraphStream is a graph handling Java library that focuses on the dynamics aspects
of graphs. Its main focus is on the modeling of dynamic interaction networks of
various sizes.

The goal of the library is to provide a way to represent graphs and work on it.
To this end, GraphStream proposes several graph classes that allow to model
directed and undirected graphs, 1-graphs or p-graphs (a.k.a. multigraphs, that
are graphs that can have several edges between two nodes).

GraphStream allows to store any kind of data attribute on the graph elements:
numbers, strings, or any object.

Moreover, in addition, GraphStream provides a way to handle the graph evolution
in time. This means handling the way nodes and edges are added and removed, and
the way data attributes may appear, disappear and evolve.
}

\subsection{Gephi}
\enquote{Gephi} ist eine Visualisierungsplatform für Graphen, welche in Java
geschrieben ist. Diese wurde 2009 an der Universität of Technologie of Compiègne
in Frankreich entwickelt. Derzeit wir die Platform vom Gehpi Consortium betreut.
Die aktuelleste Version ist 0.9.2 .

Die Bibliothek baut auf der Netbeans Platform auf. Die Program bietet dabei nur
die Basisfunktionalität an. Zusätzliche Funktionen werden durch Plug-Ins
realisiert. Die Streaming-Funktionalität besteht aus zwei Teilen, dem Client und
dem Server. Die Kommunikation erfolgt dabei über \gls{JSON}.

\foreigntextquote{english}[\cite{Bastian2009}]{
Gephi Graph Streaming is divided in different modules: The core modules, that
defines the Graph Streaming API and its implementation, the Server modules,
responsible for the HTTP REST Server, and the interface modules. 
}

\enquote{Gephi} macht über das Streaming-Modell ebenfalls keine genauen Angaben.
Da \enquote{Gephi} wie auch \enquote{graphstream-project} Events für
Knotenerzeugung und extra Angaben wie Gewichte,~\dots erlaubt, ist hier
ebenfalls von einem \enquote{Classical Streaming} Modell auszugehen.
