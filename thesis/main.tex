\documentclass[a4paper, fontsize=11pt, headsepline, listof=totoc, bibliography=totoc]{scrreprt}
\usepackage{scrlayer-scrpage}
\automark[chapter]{chapter}
\pagestyle{plain}
\usepackage{mathtools}
\usepackage{fontspec}
\defaultfontfeatures{Ligatures=TeX, Scale=MatchLowercase}
\newcommand{\euro}{€}
\usepackage[usenames, dvipsnames]{xcolor}
\usepackage{polyglossia}
\setdefaultlanguage[spelling=new]{german}
\setotherlanguage[variant=british]{english}
\usepackage[backend=biber, alldates=iso, style=iso-authoryear]{biblatex}
\addbibresource{bib/main.bib}
\usepackage[section]{placeins}
\usepackage{minted}
\usepackage{csquotes}
\renewcommand{\listoflistingscaption}{Quellcodeverzeichnis}
\usepackage{longtable, booktabs}
\usepackage{graphicx, grffile}
\usepackage[pdfencoding=unicode, pdftitle={Evaluation aktueller Bibliotheken für Stream Graph Processing},
pdfauthor={Mirko Lelansky}, pdfkeywords={{Graph}, {Stream Processing}, {Integration}, {Apache Flink}},
citecolor=black, colorlinks=true, linkcolor=black, unicode=true, urlcolor=black]{hyperref}
\usepackage[acronym, toc]{glossaries}
\newacronym{API}{API}{Application Programming Interface}
\newacronym{CRUD}{CRUD}{Create, Read, Update and Delete}
\newacronym{CRAP}{C.R.A.P.}{Change Risk Anti-Patterns}
\newacronym{DNS}{DNS}{Domain Name Services}
\newacronym{DRY}{DRY}{Don't Repeat Yourself}
\newacronym{EAR}{EAR}{Enterprise Archive}
\newacronym{EJB}{EJB}{Enterprise Java Bean}
\newacronym{HATEOAS}{HATEOAS}{Hypermedia as the Engine of Application State}
\newacronym{IDE}{IDE}{Integrated Development Environment}
\newacronym{IP}{IP}{Internet Protocol}
\newacronym{JMX}{JMX}{Java Management Extensions}
\newacronym{JPA}{JPA}{Java Persistence \acrshort{API}}
\newacronym{JSON}{JSON}{JavaScript Object Notation}
\newacronym{JVM}{JVM}{Java Virtual Machine}
\newacronym{LOC}{LOC}{Lines of Code}
\newacronym{ORM}{ORM}{Object Relational Mapping}
\newacronym{POJO}{POJO}{Plain Old Java Object}
\newacronym{PSA}{PSA}{Production Support Application}
\newacronym{REST}{REST}{Representational State Transfer}
\newacronym{RPC}{RPC}{Remote Procedure Call}
\newacronym{SOA}{SOA}{Service-Oriented Architecture}
\newacronym{SOAP}{SOAP}{Simple Object Access Protocol}
\newacronym{SOC}{SOC}{Separation of Concerns}
\newacronym[longplural=Uniquie Resources Identifiers]{URI}{URI}{Unique Resource Identifier}
\newglossaryentry{reflection}{name=Reflection, description={Reflection erlaubt
es Klassen und Objekte zur Laufzeit zu untersuchen und im begrenzten Umfang zu
modifiziern.}}
\newglossaryentry{stack}{name=Stack, description={Datenstruktur, bei der die
zuletzt eingefügten Elemente zuerst herausgenommen werden. Dieses Prinzip wird
auch als LIFO(\enquote{Last In, First Out}) bezichnet.}}

\makeglossaries
\setlength{\parskip}{\medskipamount}
%% Deutsche Kurzfassung und englisches Abstract auf eine Seite
\renewenvironment{abstract}{
    \begin{center}
    \normalfont\sectfont\nobreak\abstractname
    \end{center}
}{
    \par
}
\begin{document}
\pagenumbering{roman}
\begin{titlepage}
\begin{center}
\includegraphics[scale=1.5]{images/THB_Logo.pdf} 
\vspace{0.5cm} 

\textbf{\begin{Large}
Fachbereich Informatik und Medien
\end{Large}}
\vspace{1cm}

\textbf{\begin{Huge}Masterarbeit\end{Huge}}
\vspace{0.5cm}

\begin{Large}
Evaluation aktueller Bibliotheken für Stream Graph Processing
\end{Large}
\vspace{2cm}

\begin{tabular}{rl}
Vorgelegt von: & Mirko Lelansky\tabularnewline
am: & 30.01.2019
\end{tabular}
\vspace{1cm}

zum Erlangen des akademischen Grades
\vspace{1cm}

\textbf{\begin{LARGE}MASTER OF SCIENCE\end{LARGE}}\\
\textbf{\begin{Large}(M.Sc.)\end{Large}}
\vspace{2cm}

\begin{tabular}{rl}
Erstbetreuer: & Prof. Dr.-Ing. Sven Buchholz\tabularnewline
Zweitbetreuer: & Prof. Dr.-Ing. Susanne Busse
\end{tabular}

\end{center}
\end{titlepage}

\begin{center}
\textsf{\textbf{\begin{Large}Selbstständigkeitserklärung\end{Large}}}
\end{center}

\noindent
Hiermit erkläre ich, dass ich die vorliegende Arbeit zum Thema

\smallskip{}

\noindent
\begin{center}
\textsf{Evaluation aktueller Bibliotheken für Stream Graph Processing}
\par
\end{center}

\smallskip{}

\noindent
vollkommen selbständig verfasst und keine anderen als die angegebenen Quellen
und Hilfsmittel benutzt sowie Zitate kenntlich gemacht habe. Die Arbeit wurde
in dieser oder ähnlicher Form noch keiner anderen Prüfungsbehörde vorgelegt.

\medskip
\noindent
Brandenburg/Havel, den %Datum

\vspace{1.7cm}
\noindent
Unterschrift

\begin{center}
\textsf{\textbf{\begin{Large}Danksagung\end{Large}}}
\end{center}

Hiermit möchte ich mich bei allen bedanken, die mich bei der Anfertigung dieser
Arbeit unterstützt haben. Mein besonderer Dank gilt meinen beiden
Betreuern Prof. Dr.-Ing Sven Buchholz und Prof. Dr.-Ing. Susanne Busse. Außerdem
möchte ich meiner Familie danken, die mich während meiner Studienzeit immer
tatkräftig unterstützt hat.

\begin{abstract}
Die Arbeit beschäftigt sich mit dem Thema \enquote{Stream-Processing} von
Graphen. Ziel der Arbeit ist es die gewählten Bibliotheken hinsichtlich ihrer
Funktionalität zu analysieren. Des Weiteren sollen die praktischen Möglichkeiten
ermittelt werden.

Dazu wird in den ersten Kapiteln die Theorie für die einzelnen Aspekte gelegt.
Dabei wird zuerst die Datenstruktur eines Graphen erläutert. Anschließend wird
das Konzept \enquote{Stream-Processing} vorgestellt. Abschließend werden die
einzelnen Bibliotheken vorgestellt und miteinander verglichen.

Nachdem die theoretischen Grundlagen gelegt wurden, wird das praktische Beispiel
erläutert und für die jeweiligen Bibliotheken ein Konzept entworfen und diesen
anschließend praktisch umgesetzt.

Abschließend erfolgt eine Zusammenfassung der Arbeit und ein Ausblick auf die
möglichen nächsten Schritte.
\end{abstract}

\begin{english}
    \begin{abstract}
        The topic of the thesis is \enquote{Stream-Processing} of graphs.
        The goal of the thesis is to analyses the functionality of the selected
        libraries. That includes the research of how useable they are in practice.
        
        In the first chapters the theory for every aspect will be explained.
        First the graph datastructure is described. After that the concept and
        methods of \enquote{Stream-Processing} are explained. At the end of
        chapter the different libraries are presented and compared.
        
        After the theory, the practical example is explained. That means that
        for every library a design is created and implemented.
        
        At the end of the thesis the summary of the work and a preview of the
        next possible steps are described.
    \end{abstract}
\end{english}

\tableofcontents
\clearpage
\pagenumbering{arabic}
\pagestyle{scrheadings}
\clearpairofpagestyles
\ihead{\includegraphics[scale=0.5]{../material/images/THB_Logo_grey.pdf}}
\ohead{\headmark}
\cfoot[\pagemark]{\pagemark}
\printglossary[type=acronym, title=Abkürzungsverzeichnis, toctitle=Abkürzungsverzeichnis]
\chapter{Einleitung}
Durch den stetigen Fortschritt und der steigenden Komplexität der Anwendungen
werden große Datenmengen erzeugt. Sei es im privaten Umfeld durch die Benutzung
von Social-Media Platformen wie Facebook, Twitter,~\dots oder im gewerblichen
Umfeld durch medinische Daten, Börsendaten,~\dots . Durch die stetige Vernetzung
von Alltagsgegenständen, welches man unter dem Begriff IoT\footnote{Internet of the Things}
zusammenfassen kann, steigen diese Datenmengen nochmals sehr stark an. Dies hat
zur Folge, dass die Verarbeitung dieser Datenmengen immer komplexer wird und dies
mit nur einem einzelnen Rechner nicht mehr möglich ist. Dies wird als \gls{BigData}
bezeichnet. Bei der Verarbeitung von großen Datenmengen wird unterschieden
zwischen Batch-Processing und Stream-Processing. Beim Batch-Processing wird eine
feste Datenmenge in kleinere Einheiten unterteilt. Diese Teildaten werden dann
von mehreren Rechnern einzeln bearbeitet und anschließen die Ergebnisse wieder
kombiniert. Beim Stream-Processing gibt es keine feste Menge, sondern die Daten
kommen kontinuierlich in die Verarbeitung. Dadurch ist eine Echtzeitverarbeitung
möglich.

\begin{figure}
\centering
\includegraphics[scale=0.375]{../material/images/bitkom-lf-bigdata-2012-data_grow.jpg}
\caption{Erhöhung der Datenmenge von 1400 bis 2006 \parencite{Weber2012}}
\label{fig:data-grow}
\end{figure}

Die Daten können dabei je nach Anwendungsfall unterschiedlich strukturiert sein.
Es wird dabei unterscherschieden zwischen unstrukturierten, semi-strukturieren
und strukturierten Daten. Strukturierte Daten haben eine feste Struktur. Bei
semi-strukturieren Daten sind lediglich einzelen Bausteine definiert,
jedoch nicht wie die Daten aus den Bausteinen entstehen. Bei unstrukturierten
Daten ist lediglich die Information vorhanden, um welche Art von Daten es sich
handelt.

In der heutigen Welt spielen vor allem Graphen eine entscheidene Rolle. Da 
Unternehmen heute die Daten aus verschiedenen Quellen kombinieren wollen, um zum
Beispiel zusätzliche Werbung zu schalten. Als Beispiel jemand kauft ein Paar
Schuhe und bekommt dazu ein Angebot zu einer Tasche angezeigt, weil zum
Beispiel anderen Kunden beide Artikel gekauft haben. Ein anderes Beispiel von
Facebook ist die Liste der Personen, welche der Benutzer eventuell kennt.
Dabei werden die Informationen des Benutzers mit den Informationen der Freunde
bzw. deren Freunde kombiniert. Existiert nun eine Person, welche zum Beispiel
in die selbe Schule gegangen ist und nur Freund eines Freundes ist, wird diese
Person der Liste der Personen hinzugefügt, welche der Benutzer eventuell kennt.

\section{Motivation}
Wenn beide Welten \gls{BigData} und Graphen mit einander kombiniert werden,
ergeben sich die verschiedene Problemsituationen. Dies kommt daher, dass Graphen
komplexe Datenstrukturen sind und sich die bestehenden Konzepte und Algorithmen
vom Batch-Processing nicht so ohne weiteres aufs Stream-Processing übertragen
lassen.

%% Einsatzgebiete Graph Streaming

\section{Ziel der Arbeit}
Ziel der Arbeit ist es die gewählten Bibliotheken hinsichtlich geeigneter
Kriterien zu vergleichen und diese mit Hilfe von Beispielen praktisch zu
demonstrieren.

Dabei sollen die auftretenden Probleme bzw. Grenzen erläutert und mögliche
Lösungen bzw. Erweiterungspunkte vorgestellt werden.

\section{Aufbau der Arbeit}
Im nächsten Kapitel werden die verwendeten Technologien und Softwaresysteme
beschrieben um einen Überblick zu bekommen. Dann wird dass Design der einzelnen
Beispiele erläutert und die sich schon abzeichnenden Probleme werden beschrieben.
Danach wird das Design anhand der technischen Umsetzung dargelegt, dabei werden
die auftretenden Probleme beschrieben. Abschliend folgt eine Zusammenfassung und
ein Ausblick auf zukünfgtige Schritte.

\chapter{Theorie}
In diesem Kapitel werden die theoretischen Grundlagen für die Arbeit beschrieben.

Dabei werden zuerst die verschiedenen Graphen erläutert. Anschließend wird der
Stream-Prozess beschrieben, mit den verschiedenen Modellen.

Danach folgt eine Vorstellung der verschiedenen Bibliotheken und welche
zugrundeliegende Streaming-Modelle dort zum Einsatz kommen. 

\section{Graphen}
Ein Graph G ist eine Datenstruktur mit der folgenden Definition: $G = (V,E)$.
Dabei ist V die Menge von Knoten und E die Menge von Kanten. Jede Kante besteht
dabei aus einen Paar von V. Bei einem ungerichteten Graphen ist das Paar, welches
die Kante representiert, ein ungeordnetes Paar oder andersgesagt eine
zweielementige Menge von V. Im Gegensatz dazu ist es bei einem gerichteten
Graphen ein geordnetes Paar.

\blockquote[\cite{Aigner2015}]{
[\dots]Ein Graph $G = (V,E)$ besteht aus einer endlichen Menge V von Ecken,
einer endlichen Menge E von Kanten, und einer Vorschrift, welche jeder Kante e
genau zwei (verschiedene oder gleiche) Ecken a und b zuordnet, die wir die
Endecken von e nennen. Normalerweise sind die Endecken a und b verschieden;
ist a = b, so nennen wir e eine Schlinge bei a. Hat e die Endecken a und b, so
sagen wir, e verbindet a und b.[\dots]}

Wenn die beide Elemente der Kante identisch sind spricht man auch von einer
Schlinge. Gibt es mehrere identische Kanten spricht man von Merhfachkanten.
Diese lassen sich zusammenfassen in dem nur eine Kante dargestellt mit einer
zusätzlichen Zahl, welche die Mehrfachkanten representiert.

Es werden zwei Klassen von Graphen unterschieden. Grpahen, welche nur keine
Schlingen und Mehrfachkanten zulassen, werden als einfache Graphen bezeichnet,
sonst als Multigraphen.

\blockquote[\cite{Gurski2010}]{
[\dots]Kanten, die mit mehr als zwei Knoten inzident sind, werden Hyperkanten
genannt. Graphen mit Hyperkanten heißen Hypergraphen. Wir betrachten in diesem
Buch jedoch vorwiegend einfache Graphen, also Graphen ohne multiple Kanten und
ohne Schleifen, und auch keine Hypergraphen.}

\section{Stream-Processing}
Stream-Processing beschreibt einen Prozess, bei dem die kontinuierlich
ankommenden Daten verarbeitet bzw. transformiert werden. Ein Stream ist eine
unendliche Liste von Elementen. Die Daten werden dabei von mindestens einer
Quelle gelesen. Anschliend werden diese von mindestens einer Verarbeitungseinheit
transformiert und abschließend in mindestens ein Ziel geschrieben. Quelle und
Ziel sind dabei externe Systeme, wie Datenbanken, Messaging-Platformen,~\dots,
welche in der Regel verschieden sind.

Die Idee des Stream-Processing ist schon sehr alt und im Laufe der Zeit sind
mehrere Modelle entstanden. Jedes Modell hat dabei gewisse Vorraussetzungen
und Einschränkungen, welche direkten Einfluss auf die konkreten Umsetzungen
haben. Es gibt die Modelle \enquote{Classical Streaming}, \enquote{semi-Streaming},
\enquote{W-Stream} und \enquote{StreamSort}.

\subsection{Classical Streaming}
Das erste Modell ist das \enquote{Classical Streaming}. Dieses wurde in den
1980 von Munro und Paterson definiert. Das Modell beschreibt die Verarbeitung
eines Streams durch eine RAM-Maschine. Der Stream ist dabei eine Folge von
Zeichen aus einem definierten Alphabet, welche sequenziell verarbeitet werden
können.

Die Prameter dieses Modells sind der Speicher der RAM-Maschine in Bits und
die Anzahl der Verarbeitungsdurchläufe des Streams. Beide Parameter sind dabei
Abhängig von der Länge des Streams und sollen möglichst klein im Verhältnis zur
Länge des Streams sein.

\foreignblockquote{english}[\cite{Ribichini2007}]{
In classical streaming, input data can be accessed sequentially in the form
of a data stream, and need to be processed using a working memory that
is small compared to the length of the stream. The main parameters of the
model are the number p of sequential passes over the data and the size s of the
working memory (in bits). [\dots]
}

Das Modell macht dabei keine Aussagen über die Laufzeit des Algorithmus. Diese
spielt für viele Probleme jedoch eine Rolle, deshalb wurde das
\enquote{semi-Streaming} Modell entwickelt.

\foreignblockquote{english}[\cite{Ribichini2007}]{
Notice that our definition imposes no restrictions on the amount
of computation performed by the RAM machine, as it is often the case when
dealing with external memory models, where it is assumed that I/O operations
take orders of magnitude longer than internal memory operations. There
are, however, applications in which the per-item processing time (average,
maximum) is a significant parameter that should be taken into account.
}

\subsection{semi-Streaming}
Das \enquote{semi-Streaming} Modell ist ein vereinfachtes Modell des
\enquote{Classical Streaming} Modells. Im Gegensatz zum \enquote{Classical Streaming}
wird hier festgelegt, das die Anzahl der Durchläufe beschränkt wird auf eins bzw.
einige wenige. Des Weiteren wird die Speichergröße der RAM-Maschine ebenfalls
auf beschränkt auf die logarithmische-Komplexitätsklasse.

\foreignblockquote{english}[\cite{Ribichini2007}]{
In particular, some recent papers show that several graph problems can be
solved with one or few passes in the Semi-streaming model [53] where the
working memory size is $O(n · \log n)$ for an input graph with n vertices
(or even $O(n^{1 + \epsilon})$, with $\epsilon < 1$, for applications like
spanners, for which linear memory in the number of vertices is provably not
sufficient): [\dots].
}

Diese Beschränkungen sorgen dafür, dass die Laufzeit des Algorithmus verbessert
wird. Jedoch haben diese Beschränkungen auch den Nachteil, dass sie nur
näherungsweise Ergebnisse liefern.

\foreignblockquote{english}[\cite{Ribichini2007}]{
Despite the heavy restrictions of the classical streaming model, major
success has been achieved for several data sketching and statistics problems,
e.g., approximate frequency moments [7], histogram maintenance [58],
L\textsuperscript{1} difference [55], where $O(1)$ passes and polylogarithmic
working space have proven enough to find approximate solutions (see also the
bibliographies in [14, 88]).
}

\subsection{W-Stream}
Das \enquote{W-Stream} Modell oder \enquote{Write-Stream} Modell ist eine
Erweiterung des \enquote{Classical Streaming}. Dabei wird jedes gelesene Element
des Streams nach der Verarbeitung in einen Ausgabe-Stream geschrieben. Die
Elemente können dabei bevor sie in den Ausgabe-Stream landen, verändert werden.
Wenn die Elemente nicht verändert werden, handelt es sich um einen speziellen
\enquote{W-Strem} nämlich den \enquote{Classical Stream}.

\foreignblockquote{english}[\cite{Ribichini2007}]{
In the W-Stream model [97], a streaming pass, while reading data from
the input stream and processing them in the working memory, produces items
that are sequentially appended to an output stream. [\dots]
Clearly, $\text{Stream}(p, s) \subseteq \text{W-Stream}(p, s)$ since at every
s/w-pass the output stream may simply consist of a copy of the input stream.
}

\subsection{StreamSort}
Das \enquote{StreamSort} Modell ist eine Erweiterung zum \enquote{W-Stream}.
Dabei schließt sich nach dem Schreibprozess noch ein Sortierungsprozess an. Dabei
werden die Elemente noch einer vorgegebenen globalen Sortierungsreihenfolge
sortiert.

\foreignblockquote{english}[\cite{Ribichini2007}]{
[\dots] This model extends classical streaming in two ways the ability to write
intermediate temporary streams and the ability to reorder them at each pass for
free. [\dots]
}

\section{Graph-Streaming Bibliotheken}
Graph-Streaming Bibliotheken sind Bibliotheken, welche die beiden Welten \gls{BigData}
und Graphen miteinander kombinieren wollen. Dabei geht es darum große Graphen
effizient und möglichst in Echtzeit mit Hilfe von Streams zu verarbeiten. Die
bedeutet, dass die Daten, welche von einem Stream verarbeitet werden, keine
einfachen Zeichenketten mehr sind, sondern zum Beispiel eine Kante des Graphens.
Je nach dem welches Streaming-Modell zum Einsatz kommt, können auch andere Daten
hinzu kommen. In dieser Arbeit werden die Bibliotheken \enquote{gelly-streaming},
\enquote{graphstream-project} und \enquote{Gephi} verglichen.

\subsection{gelly-streaming}
\enquote{gelly-streaming} ist eine Graph-Streaming Bibliothek für Apache Flink,
welche von Vasiliki Kalavri am KTH in Schweden im Jahr 2015 entwickelt wurde.
Apache Flink eine verteielte Platform für Batch- und Stream-Processing. Der Kern
von Apache Flink ist die Streaming-Engine, welche die eigentliche Verarbeitung
der Daten vornimmt. Alle \glspl{API} und Bibliotheken bauen auf der Engine auf.
Apache Flink wurde an der TU Berlin entwickelt und ist jetzt ein Apache Projekt.
Apache Flink ist in der Programmiersprache Java geschrieben. Es existiert jedoch
auch eine \gls{API} für Scala, welche nur ein Wrapper für Java darstellt. Dadurch
können Entwickler ihre Programme in Java oder Scala schreiben. Für alle anderen
Java-Alternativen müssen die benötigten Laufzeitbibliotheken mitgeliefert
werden. Die aktuelle Version von Flink ist 1.6.0 .

Bei der Bibliothek \enquote{gelly-streaming} kommt ein semi-streaming Verfahren
zum Einsatz, da die Bibliothek für Apache Flink entwickelt wurde. Dabei wird
jeweils immer eine Kante mit den beiden Knoten gespeichert. Jedoch keine
Zusatzinformationen, wie zum Beispiel ob die Kante hinzugefügt oder entfernt
wurde. Dies muss vom Anwender sichergestellt werden. \enquote{gelly-streaming}
ist ebenfalls in Java geschrieben und wird vorrangig von Studenten von
Vasiliki Kalavri weiterentwickelt. Allerdings ist anzumerken, dass die Bibliothek
zum Zeitpunkt der Arbeit noch im experimentellen Stadium ist und deshalb noch
keine Alpha/Beta/Finale-Versionen zur Verfügung stehen.

%% Bild mit gelly-streaming

\subsection{graphstream-project}
\enquote{graphstream-project} ist eine Java-Bibliothek für Graph-Streaming. Die
Bibiliothek wurde 2009 von der Gruppe RI\textsubscript{2}C-Gruppe am LITIS ein
Zusammenschluss von mehreren Universitäten und Firmen entwickelt. Derzeit wird
die Bibliothek von der Universität Le Havre weiterentwickelt. Die aktuelleste
Version ist 1.3 .


%% Streaming-Modell
%% Architektur

\subsection{Gephi}
\enquote{Gephi} ist eine Visualisierungsplatform für Graphen, welche in Java
geschrieben ist. Diese wurde 2009 an der Universität of Technologie of Compiègne
in Frankreich entwickelt. Derzeit wir die Platform vom Gehpi Consortium betreut.
Die aktuelleste Version ist 0.9.2 .

%% Streaming-Modell
%% Architektur

\chapter{Design}
In diesem Kapitel geht es darum die Vor- und Nachteile der unterschiedlichen
Bibliotheken anhand eines praktischen Beispieles zu zeigen. Dabei wird
\enquote{gelly-streaming} als Referenz benutzt. Zunächst wird dabei allgemein
der Ablauf des Beispiel beschrieben. Anschließend wir die Architektur der
jeweiligen Umsetzungen beschrieben und schon auftretende Probleme beschrieben.
Abschließend werden mögliche Lösungsstrategien für die Probleme erläutert.

\section{Analyse des Beispieles}
In dem Beispiel geht es darum zu überprüfen, ob ein Graph bipartite ist oder
nicht. Dies bedeutet, ob ein Graph zweifarbig färbbar ist. Um dies für einen
existierenden Graphen zu lösen, gibt es verschiedene Algorithmen. Die allgemeine
Vorgehensweise ist dabei ähnlich.

Am Begin sind alle Knoten ungefärbt. Dann werden zwei Farben gewählt.
Anschließend wird ein Startknoten gewählt und dieser in einer der Farben markiert.
Dann werden die unmarkierten Nachbarknoten in der anderen Farbe markiert. Danach
werden deren unmarkierte Nachbarknoten wieder in der ersten Farbe markiert, usw.
bis alle Knoten makiert sind. Ein Graph ist dann bipartite, wenn alle Knoten so
markiert sind, dass keine benachbarten Knoten die selbe Farbe haben. Diese
Vorgehensweise kann jeder selbst für kleine Graphen selbst durchführen. Je nachdem
welche Informationen über den zu überprüfenden Graphen existieren, kann der Ablauf
auch verkürzt werden. Wenn der Graph nämlich einen Kreis von ungerader Länge
enthällt ist, kann der Graph nie bipartite sein unabhängig vom gewählten
Startknoten. Die Abbildung \ref{fig:bipartite-six-nodes} zeigt einen bipartiten
Graphen mit einen Kreis der Länge sechs. Während dessen die Abbildung \ref{fig:bipartite-five-nodes}
einen nicht bipartiten Graphen zeigt. Denn der Kreis hat eine ungerade Länge und
lässt sich damit nie mit zwei Farben färben egal welcher Startknoten gewählt
wurde. 

\begin{figure}
\centering
\includegraphics[scale=0.5]{../material/images/bipartite-graph-six-nodes.jpg}
\caption{bipartiter Graph mit Kreis der Länge sechs \cite{GeeksforGeeks2018}}
\label{fig:bipartite-six-nodes}
\end{figure}

\begin{figure}
\centering
\includegraphics[scale=0.5]{../material/images/bipartite-graph-five-nodes.jpg}
\caption{nicht bipartiter Graph mit Kreis der Länge fünf \cite{GeeksforGeeks2018}}
\label{fig:bipartite-five-nodes}
\end{figure}


Wie läuft das Beispiel nun konkret ab. Zunächst werden die Daten eingelesen. Dies
erfolgt zur Vereinfachung über eine Datei. Dies kann jedoch jederzeit auf eine
alternative Eingabe umgestellt werden zum Beispiel um die Daten von einem
Messaging-System wie Apache Kafka einlesen zu können. Dabei ist zu beachten,
dass alle Bibliotheken, verschiedene Eingabeformate haben, welche in der
Architekturbeschreibung konkret beschrieben werden. Anschließend werden die
Daten an eine beliebige Ausgabe gesendet. Dies ist in unserem Fall einfach die
Konsole. In einer Produktionsumgebung ist dies gerade für Apache Flink
problematisch besonders, wenn es verteilt auf mehrere Rechnern betrieben wird.

\subsection{Architektur}

\chapter{Realisierung}
In diesem Kapitel geht es darum, die wichtigsten Punkte der Realisierungen
zu beschreiben. Die Bibliothek \enquote{gelly-streaming} wird dabei wieder als
Referenz benutzt. Anschließend werden die Realisierung und deren Benutzbarkeit
analysiert und mögliche Lösungsen beschrieben.

\section{Analyse der Umsetzung}
\subsection{Umsetzung der Referenz-Implementierung}
\subsection{Umsetzung von \enquote{graphstream-project}}
\subsection{Umsetzung von \enquote{Gephi}}
\section{Bewertung der Umsetzungen}

\chapter{Zusammenfassung}
In diesem Kapitel wird die Arbeit kurz zusammengefasst und ein Überblick über
die nächsten Schritte gegeben.

\section{Fazit}
In dieser Arbeit wurde gezeigt, wie weit der Entwicklungsstand der gewählten
Graph-Bibliotheken ist. Dazu war es notwendig sich erst einmal über die
unterschiedlichen Konzepte zu informieren. Denn sowohl \enquote{Stream-Processing},
als auch Graphen sind für sich genommen schon sehr komplexe und teils neue Themen.
Dies gilt besonders in Bezug auf das Thema \gls{BigData}. Die Kombination von
beiden Themen ist dementsprechend nochmals komplexer, denn Konzepte, welche in
der eigenen Welt funktionieren, müssen in der kombinierten Welt nicht unbedingt
funktionieren. Denn es gibt verschiedene \enquote{Stream-Processing} Konzepte und
ein Graph besteht ebenfalls aus mehreren Konzepten. Dadurch ist auch klar, warum
es nicht die eine Bibliothek gibt, weil jede Bibliothek unterschiedliche Konzepte
gewählt hat und diese testen muss, ob sie für einen produktiven Einsatz geeignet
sind.

Zum Zeitpunkt der Arbeit waren alle Bibliothek noch in einem experimentellen
Status. Dies wird vermutlich noch eine Weile so bleiben. Denn obwohl alle
Bibliothek über gute Ansätze verfügen, mangelt es vor allem noch an ausreichendem
Einsatzgebieten und den daraus resultierenden Testfällen, Daten und Beispielen.
Die derzeit vorhandenen Beispiele gehen mehr oder weniger nicht über ein normales
\enquote{Hello World} hinaus. Obwohl \enquote{gelly-streaming} durch Apache Flink
schon über einen guten Unterbau verfügt. Des Weiteren scheint die Entwicklung bei
allen mehr oder weniger zu stagnieren. Dies kann natürliche unterschiedliche
Ursachen haben. Zunächst einmal alle Bibliothek wurden als Pionierarbeit an
Universitäten entwickelt. Diese ist schon mal ein enormer Aufwand, der im
Allgemeinen auch nicht entlohnt wird. Ausgehend von dieser Pionierarbeit dann
eine Firma zu gründen bzw. eine Firma zu finden, welche dies weiterentwickelt ist
nicht gerade einfach, da die Projekte meistens noch nicht wirklich ausgereift
sind. Die Firmen müssen jedoch ihre Finanzen im Blick behalten, denn sie haben
ja Angestellte und diese wollen auch noch in einigen Jahren ihren Lohn
bekommen. Ein Projekt zu finanzieren, welches dann vielleicht nach einiger
Entwicklungszeit von den Projektbeteiligten eingestellt werden muss ist, ist da
selten zu verkraften. Bei kleinen hausinternen Projekten ist, dies meistens
noch eher zu verkraften. Auf der anderen Seite, wenn es sich um eine
revolutionäre Idee handelt, welche einzigartig und von den Firmen dringen gesucht
wird, wird dies selten als Open-Source-Projekt zu Ende gebracht. Denn diese
Projekte sind meistens die Einstiege in neue Jobs. Ein weiterer Punkt ist natürlich,
dass die aktiven Entwickler sich anderen Projekten bzw. Aufgaben zu wenden können.
Dies wird gerade auch bei \enquote{gelly-streaming} klar, wo derzeit keine bis
nur eine minimale Entwicklung stattfindet, der Unterbau Apache Flink wird
im Gegensatz dazu jedoch sehr stark weiterentwickelt. Wie die weitere Entwicklung
von Apache Flink weitergeht, lässt sich nicht vorhersagen. Dies gilt ins
besondere, da die Firma \enquote{data Artisan}, welche die Entwicklung von
Apache Flink betreibt, von \enquote{Alibaba} gekauft wurde.

Das Beispiel konnte nicht mit allen Bibliotheken praktisch umgesetzt werden. Dies
hängt vor allem mit der zum Teil schlechten Infrastruktur und dem experimentellen
Status zusammen, denn die Entwürfe für die einzelnen Bibliotheken wurden ja
entwickelt. Alle Bibliotheken nutzen sehr verschiedene Ansätze und haben noch
einige Probleme bzw. müssen sich noch weiter bewähren.

Deshalb gibt es auch keinen klaren Favoriten. Wenn jedoch Apache Flink in die
Betrachtung mit einfließt, dann hat \enquote{gelly-streaming} schon einen klaren
Vorteil. Zumal Apache Flink mit \enquote{Gelly} schon eine Graph-Bibliothek für
\enquote{Batch-Processing} hat und es angedacht ist, dass diese mit
\enquote{gelly-streaming} zusammengeführt wird. Die Bibliothek \enquote{Table API \& SQL}
baut sowohl auf der \enquote{Stream-Processing}- wie auch auf der
\enquote{Batch-Processing}-\gls{API} von Apache Flink auf. Es ist jedoch klar
anzumerken, dass keine der hier getesteten Bibliotheken für einen produktiven
Einsatz geeignet sind.

\section{Ausblick}
Für die nächsten Entwicklungsschritte bieten sich mehrere Möglichkeiten an. Zum
einen könnte die gesamte Infrastruktur von \enquote{Gephi} aktualisiert werden.
Dadurch würde eine weitere Bibliothek zum Testen hinzukommen bzw. die zugrunde
liegenden Ideen könnten überprüft und gegebenenfalls für andere Projekte benutzt
werden.

Eine andere Möglichkeiten besteht darin, für die beiden anderen Bibliotheken ein
größeres Beispiel zu definieren und dieses dann praktisch umzusetzen. Um
anschließend genau zu wissen, welche Bibliothek für welchen Einsatzzweck geeignet
ist und welche nicht weiter entwickelt werden sollte. Eine direkte Weiterentwicklung
von \enquote{gelly-streaming} ist jedoch abzuraten, da diese eine direkte Beziehung
zu Apache Flink hat und es ist deshalb anzuraten sich mit den Entwicklern von
Apache Flink auseinander zu setzen.

\printglossary
\listoffigures
\listoflistings
\printbibliography
\end{document}
