\documentclass[a4paper, fontsize=11pt]{beamer}
\usetheme{Berlin}
\usepackage{mathtools}
\usepackage{fontspec}
\defaultfontfeatures{Ligatures=TeX, Scale=MatchLowercase}
\newcommand{\euro}{€}
\usepackage{polyglossia}
\setdefaultlanguage[spelling=new]{german}
\setotherlanguage[variant=british]{english}
\usepackage{csquotes}
\usepackage[backend=biber, style=iso-authoryear]{biblatex}
\usepackage[section]{placeins}
\usepackage{minted}
\renewcommand{\listoflistingscaption}{Quellcodeverzeichnis}
\usepackage{longtable, booktabs}
\usepackage{graphicx, grffile}
\usepackage{hyperref}
\usepackage[acronym, toc]{glossaries}
\setlength{\parskip}{\medskipamount}
\begin{document}
\begin{frame}
\author{Mirko Lelansky}
\title{Kolloquium: Evaluation aktueller Bibliotheken für Stream Graph Processing}
\date{28.02.2019}
\maketitle
\end{frame}

\begin{frame}
\tableofcontents
\end{frame}

\section{Einführung}
\begin{frame}{Hintergründe}
\begin{itemize}
\item stetige Datenerzeugung
\item Verarbeitung der Daten um Mehrwert zu generieren
\item zum Beispiel welche Produkte liegen aktuell im Trend
\item schnelle Ergebnisse gefordert
\item durch Fortschritt und Entwicklung riesiger Datensprung
\item unterschiedlich strukturierte Daten
\end{itemize}
\end{frame}

\begin{frame}{Graphen}
\begin{itemize}
\item komplexe Datenstruktur $ G = (V,E)$
\item Liste von Knoten und Liste von Kanten
\item Kante ist ein un- bzw. geordnetes Paar von Kanten
\item Weg ist eine Menge von zusammenhängen Kanten
\item mehrere Spezialfälle
\end{itemize}
\end{frame}

\begin{frame}{Problemstellung}
\begin{itemize}
\item unterschiedliche Anforderungen
\item große Datenmengen
\item schnelle Antwortzeiten
\item komplexe Datenstrukturen
\item Darstellung von Graphen für Verarbeitung
\end{itemize}
\end{frame}

\begin{frame}{Aufgabenstellung}
\begin{itemize}
\item Bibliotheken suchen und wählen
\item Bibliotheken anhand ihrere Eigenschaften vergleichen
\item Funktionen demonstrieren
\item mögliche Einstiegspunkte bzw. Erweiterungspunkte herausarbeiten
\end{itemize}
\end{frame}

\begin{frame}{Motivation}
\begin{itemize}
\item tiefer in die Streaming-Materie einsteigen
\item Verständnis von Graphen erweitern
\item alternative BigData-Anwendung kennenlernen
\end{itemize}
\end{frame}

\section{Analyse}
\section{Umsetzung}
\section{Zusammenfassung}
\end{document}
